% Apresentação em Slides - Inteligência Empresarial
% Dashboard de Saúde Financeira de Empresas Tecnológicas

\documentclass[10pt]{beamer}

% Tema e configuração
\usetheme{Madrid}
\usecolortheme{default}
\setbeamertemplate{navigation symbols}{}
\setbeamertemplate{footline}[frame number]

% Pacotes
\usepackage[utf8]{inputenc}
\usepackage[portuguese]{babel}
\usepackage{graphicx}
\usepackage{booktabs}
\usepackage{listings}
\usepackage{xcolor}
\usepackage{tikz}
\usepackage{hyperref}

% Configuração de cores
\definecolor{darkblue}{RGB}{30,58,95}
\definecolor{lightblue}{RGB}{103,137,177}
\definecolor{green}{RGB}{56,171,63}
\definecolor{red}{RGB}{235,87,87}
\definecolor{yellow}{RGB}{242,201,76}

% Configuração de código
\lstset{
    basicstyle=\ttfamily\footnotesize,
    breaklines=true,
    frame=single,
    language=Python
}

% Informações do título
\title[Dashboard Saúde Financeira]{Sistema de Análise de Saúde Financeira}
\subtitle{Dashboard Interativo para Empresas Tecnológicas}
\author{António Sousa}
\institute{Faculdade de Economia da Universidade de Coimbra}
\date{\today}

\begin{document}

% ========== SLIDE 1: TÍTULO ==========
\begin{frame}
\titlepage
\end{frame}

% ========== SLIDE 2: AGENDA ==========
\begin{frame}{Agenda}
\tableofcontents
\end{frame}

% ========== SECÇÃO 1: INTRODUÇÃO ==========
\section{Apresentação do Projeto}

\begin{frame}{Contexto e Problema}
\begin{block}{Problema Identificado}
\begin{itemize}
    \item Dificuldade em avaliar rapidamente a saúde financeira de empresas
    \item Dados financeiros dispersos e difíceis de interpretar
    \item Falta de ferramentas intuitivas para análise comparativa
    \item Necessidade de métricas padronizadas e automatizadas
\end{itemize}
\end{block}

\vspace{0.5cm}

\begin{block}{Objetivo do Projeto}
Desenvolver um \textbf{sistema automatizado} que:
\begin{itemize}
    \item Extrai dados financeiros de fontes públicas (Yahoo Finance)
    \item Calcula indicadores-chave de performance (KPI)
    \item Apresenta análises visuais através de dashboard interativo
    \item Permite decisões informadas sobre saúde empresarial
\end{itemize}
\end{block}
\end{frame}

\begin{frame}{Empresas Analisadas}
\begin{block}{Top 10 Empresas Tecnológicas}
\begin{columns}[T]
\begin{column}{0.5\textwidth}
\begin{itemize}
    \item \textbf{AAPL} - Apple Inc.
    \item \textbf{MSFT} - Microsoft Corporation
    \item \textbf{GOOGL} - Alphabet Inc.
    \item \textbf{AMZN} - Amazon.com Inc.
    \item \textbf{NVDA} - NVIDIA Corporation
\end{itemize}
\end{column}
\begin{column}{0.5\textwidth}
\begin{itemize}
    \item \textbf{META} - Meta Platforms Inc.
    \item \textbf{TSLA} - Tesla Inc.
    \item \textbf{AVGO} - Broadcom Inc.
    \item \textbf{ASML} - ASML Holding N.V.
    \item \textbf{NFLX} - Netflix Inc.
\end{itemize}
\end{column}
\end{columns}
\end{block}

\vspace{0.3cm}

\begin{alertblock}{Setor}
Todas as empresas pertencem ao setor \textbf{Technology}, permitindo comparações diretas.
\end{alertblock}
\end{frame}

% ========== SECÇÃO 2: ARQUITETURA ==========
\section{Arquitetura do Sistema}

\begin{frame}{Arquitetura Medallion (Bronze-Silver-Gold)}
\begin{center}
\begin{tikzpicture}[scale=0.8, every node/.style={transform shape}]
    % Bronze Layer
    \draw[fill=yellow!30, thick] (0,0) rectangle (3,2);
    \node at (1.5,1.5) {\textbf{BRONZE}};
    \node[align=center] at (1.5,0.8) {Dados Brutos\\Yahoo Finance};
    \node[align=center, font=\tiny] at (1.5,0.2) {yfinance API};
    
    % Arrow 1
    \draw[->, thick] (3.2,1) -- (4.8,1);
    \node[above, font=\tiny] at (4,1.2) {Limpeza};
    
    % Silver Layer
    \draw[fill=blue!20, thick] (5,0) rectangle (8,2);
    \node at (6.5,1.5) {\textbf{SILVER}};
    \node[align=center] at (6.5,0.8) {Dados\\Padronizados};
    \node[align=center, font=\tiny] at (6.5,0.2) {PostgreSQL};
    
    % Arrow 2
    \draw[->, thick] (8.2,1) -- (9.8,1);
    \node[above, font=\tiny] at (9,1.2) {KPI};
    
    % Gold Layer
    \draw[fill=green!30, thick] (10,0) rectangle (13,2);
    \node at (11.5,1.5) {\textbf{GOLD}};
    \node[align=center] at (11.5,0.8) {Analytics \&\\Métricas};
    \node[align=center, font=\tiny] at (11.5,0.2) {Health Score};
    
    % Dashboard
    \draw[fill=lightblue!30, thick] (5,-1.5) rectangle (8,-0.5);
    \node at (6.5,-1) {\textbf{DASHBOARD}};
    
    \draw[->, thick] (11.5,-0.2) -- (11.5,-1) -- (8,-1);
\end{tikzpicture}
\end{center}

\begin{block}{Camadas do Pipeline}
\begin{itemize}
    \item \textbf{Bronze}: Extração de dados brutos via API yfinance
    \item \textbf{Silver}: Transformação e normalização dos dados
    \item \textbf{Gold}: Cálculo de KPIs e métricas de saúde financeira
    \item \textbf{Dashboard}: Visualização interativa (Streamlit)
\end{itemize}
\end{block}
\end{frame}

% ========== SECÇÃO 3: KPI ==========
\section{Indicadores de Performance (KPI)}

\begin{frame}{KPIs - Liquidez}
\begin{block}{Indicadores de Liquidez}
Medem a capacidade da empresa pagar obrigações de curto prazo.
\end{block}

\begin{table}[h]
\centering
\small
\begin{tabular}{lll}
\toprule
\textbf{Indicador} & \textbf{Fórmula} & \textbf{Benchmark} \\
\midrule
Current Ratio & $\frac{\text{Ativos Correntes}}{\text{Passivos Correntes}}$ & $\geq 1.5$ \\[0.3cm]
Quick Ratio & $\frac{\text{Ativos Líquidos}}{\text{Passivos Correntes}}$ & $\geq 1.0$ \\[0.3cm]
Cash Ratio & $\frac{\text{Caixa}}{\text{Passivos Correntes}}$ & $\geq 0.25$ \\
\bottomrule
\end{tabular}
\end{table}

\begin{alertblock}{Interpretação}
\begin{itemize}
    \item \textcolor{green}{\textbf{Alto}}: Empresa tem capacidade de pagar dívidas de curto prazo
    \item \textcolor{red}{\textbf{Baixo}}: Risco de insolvência no curto prazo
\end{itemize}
\end{alertblock}
\end{frame}

\begin{frame}{KPIs - Rentabilidade}
\begin{block}{Indicadores de Rentabilidade}
Avaliam a capacidade da empresa gerar lucro.
\end{block}

\begin{table}[h]
\centering
\small
\begin{tabular}{lll}
\toprule
\textbf{Indicador} & \textbf{Fórmula} & \textbf{Benchmark} \\
\midrule
Gross Margin & $\frac{\text{Lucro Bruto}}{\text{Receita}}$ & $\geq 30\%$ \\[0.3cm]
Net Margin & $\frac{\text{Lucro Líquido}}{\text{Receita}}$ & $\geq 10\%$ \\[0.3cm]
ROE & $\frac{\text{Lucro Líquido}}{\text{Equity}}$ & $\geq 15\%$ \\[0.3cm]
ROA & $\frac{\text{Lucro Líquido}}{\text{Ativos}}$ & $\geq 5\%$ \\
\bottomrule
\end{tabular}
\end{table}

\begin{alertblock}{Interpretação}
Margens elevadas indicam \textbf{forte poder de precificação} e controlo de custos.
\end{alertblock}
\end{frame}

\begin{frame}{KPIs - Alavancagem}
\begin{block}{Indicadores de Alavancagem}
Medem o nível de endividamento da empresa.
\end{block}

\begin{table}[h]
\centering
\small
\begin{tabular}{lll}
\toprule
\textbf{Indicador} & \textbf{Fórmula} & \textbf{Benchmark} \\
\midrule
Debt-to-Equity & $\frac{\text{Dívida Total}}{\text{Equity}}$ & $\leq 1.0$ \\[0.3cm]
Debt-to-Assets & $\frac{\text{Dívida Total}}{\text{Ativos}}$ & $\leq 50\%$ \\
\bottomrule
\end{tabular}
\end{table}

\begin{alertblock}{Interpretação}
\begin{itemize}
    \item \textcolor{green}{\textbf{Baixo}}: Estrutura de capital conservadora, menor risco
    \item \textcolor{red}{\textbf{Alto}}: Alta alavancagem, risco financeiro elevado
\end{itemize}
\end{alertblock}
\end{frame}

\begin{frame}{KPIs - Cash Flow}
\begin{block}{Indicadores de Fluxo de Caixa}
Avaliam a geração de caixa operacional.
\end{block}

\begin{table}[h]
\centering
\small
\begin{tabular}{lll}
\toprule
\textbf{Indicador} & \textbf{Fórmula} & \textbf{Benchmark} \\
\midrule
OCF Ratio & $\frac{\text{CF Operacional}}{\text{Passivos Correntes}}$ & $\geq 1.0$ \\[0.3cm]
FCF Margin & $\frac{\text{Free Cash Flow}}{\text{Receita}}$ & $\geq 15\%$ \\
\bottomrule
\end{tabular}
\end{table}

\begin{alertblock}{Importância}
Free Cash Flow é crucial - representa dinheiro \textbf{disponível} para:
\begin{itemize}
    \item Dividendos
    \item Recompra de ações
    \item Investimentos
    \item Redução de dívida
\end{itemize}
\end{alertblock}
\end{frame}

\begin{frame}{Health Score (0-100)}
\begin{block}{Cálculo do Health Score}
Score composto baseado em 4 dimensões:
\end{block}

\begin{table}[h]
\centering
\small
\begin{tabular}{lc}
\toprule
\textbf{Dimensão} & \textbf{Peso} \\
\midrule
Liquidez & 20 pontos \\
Rentabilidade & 30 pontos \\
Alavancagem & 25 pontos \\
Cash Flow & 25 pontos \\
\midrule
\textbf{Total} & \textbf{100 pontos} \\
\bottomrule
\end{tabular}
\end{table}

\begin{block}{Classificação}
\begin{itemize}
    \item \textcolor{green}{\textbf{80-100}} - Excellent: Posição financeira forte
    \item \textcolor{green!70!black}{\textbf{65-79}} - Good: Finanças saudáveis
    \item \textcolor{yellow!80!black}{\textbf{50-64}} - Fair: Algumas áreas precisam atenção
    \item \textcolor{orange}{\textbf{35-49}} - Concerning: Múltiplos sinais de alerta
    \item \textcolor{red}{\textbf{0-34}} - Poor: Dificuldades financeiras sérias
\end{itemize}
\end{block}
\end{frame}

% ========== SECÇÃO 4: MODELO DE DADOS ==========
\section{Modelo de Dados}

\begin{frame}{Modelo Entidade-Relacionamento (ER)}
\begin{center}
\begin{tikzpicture}[scale=0.7, every node/.style={transform shape}]
    % Entities
    \draw[thick, fill=lightblue!30] (0,4) rectangle (3,5.5);
    \node at (1.5,5) {\textbf{COMPANY}};
    \node[align=left, font=\tiny] at (1.5,4.4) {ticker (PK)\\company\_name\\sector};
    
    \draw[thick, fill=yellow!30] (6,6) rectangle (9,7.5);
    \node at (7.5,7) {\textbf{INCOME\_STMT}};
    \node[align=left, font=\tiny] at (7.5,6.4) {id (PK)\\ticker (FK)\\fiscal\_year\\revenue\\net\_income};
    
    \draw[thick, fill=yellow!30] (6,3.5) rectangle (9,5);
    \node at (7.5,4.5) {\textbf{BALANCE\_SHEET}};
    \node[align=left, font=\tiny] at (7.5,3.9) {id (PK)\\ticker (FK)\\fiscal\_year\\total\_assets\\equity};
    
    \draw[thick, fill=yellow!30] (6,1) rectangle (9,2.5);
    \node at (7.5,2) {\textbf{CASH\_FLOW}};
    \node[align=left, font=\tiny] at (7.5,1.4) {id (PK)\\ticker (FK)\\fiscal\_year\\operating\_cf\\fcf};
    
    \draw[thick, fill=green!30] (11.5,3.5) rectangle (14.5,5);
    \node at (13,4.5) {\textbf{HEALTH}};
    \node[align=left, font=\tiny] at (13,3.9) {id (PK)\\ticker (FK)\\fiscal\_year\\health\_score\\status};
    
    % Relationships
    \draw[->, thick] (3,4.75) -- (6,6.5);
    \draw[->, thick] (3,4.75) -- (6,4.25);
    \draw[->, thick] (3,4.75) -- (6,1.75);
    \draw[->, thick] (9,4.25) -- (11.5,4.25);
    
    \node[above, font=\tiny] at (4.5,5.8) {1:N};
    \node[above, font=\tiny] at (4.5,4.5) {1:N};
    \node[above, font=\tiny] at (4.5,3) {1:N};
    \node[above, font=\tiny] at (10.2,4.5) {1:1};
\end{tikzpicture}
\end{center}

\begin{alertblock}{Nota}
Cada empresa (\textit{ticker}) possui múltiplos anos fiscais de dados financeiros.
\end{alertblock}
\end{frame}

\begin{frame}{Modelo Relacional - Tabelas Silver}
\begin{block}{silver\_companies}
\texttt{ticker (PK), company\_name, sector}
\end{block}

\begin{block}{silver\_income\_statement}
\texttt{id (PK), ticker (FK), fiscal\_year, revenue, cost\_of\_revenue,\\
gross\_profit, operating\_income, net\_income, ebitda, eps\_basic}
\end{block}

\begin{block}{silver\_balance\_sheet}
\texttt{id (PK), ticker (FK), fiscal\_year, total\_assets, total\_liabilities,\\
total\_equity, current\_assets, current\_liabilities, cash\_and\_equivalents,\\
total\_debt}
\end{block}

\begin{block}{silver\_cash\_flow}
\texttt{id (PK), ticker (FK), fiscal\_year, operating\_cash\_flow,\\
investing\_cash\_flow, financing\_cash\_flow, free\_cash\_flow,\\
capital\_expenditures}
\end{block}
\end{frame}

\begin{frame}{Modelo Relacional - Tabelas Gold}
\begin{block}{gold\_financial\_health}
\texttt{id (PK), ticker (FK), company\_name, fiscal\_year,\\
current\_ratio, quick\_ratio, cash\_ratio,\\
gross\_margin, operating\_margin, net\_margin, roe, roa,\\
debt\_to\_equity, debt\_to\_assets,\\
operating\_cash\_flow\_ratio, free\_cash\_flow\_margin,\\
health\_score, health\_status, analysis\_notes}
\end{block}

\begin{block}{gold\_kpi\_dashboard}
\texttt{id (PK), ticker (FK), company\_name, fiscal\_year,\\
revenue, revenue\_growth, net\_income, profit\_growth,\\
total\_assets, total\_debt, free\_cash\_flow,\\
current\_ratio, debt\_to\_equity, net\_margin, roe,\\
health\_score, health\_status,\\
revenue\_rank, profit\_rank, health\_rank}
\end{block}
\end{frame}

\begin{frame}{Tecnologias Utilizadas}
\begin{columns}[T]
\begin{column}{0.5\textwidth}
\begin{block}{Backend \& Database}
\begin{itemize}
    \item \textbf{Python 3.12}
    \item \textbf{PostgreSQL} - Base de dados relacional
    \item \textbf{yfinance} - API Yahoo Finance
    \item \textbf{pandas} - Manipulação de dados
    \item \textbf{psycopg2} - Connector PostgreSQL
\end{itemize}
\end{block}
\end{column}

\begin{column}{0.5\textwidth}
\begin{block}{Frontend \& Visualização}
\begin{itemize}
    \item \textbf{Streamlit} - Dashboard web
    \item \textbf{Plotly} - Gráficos interativos
    \item \textbf{Excel} - Exports para BI
    \item \textbf{Power BI} - Análise avançada
\end{itemize}
\end{block}
\end{column}
\end{columns}

\vspace{0.5cm}

\begin{alertblock}{Estrutura do Projeto}
\begin{itemize}
    \item \texttt{src/data\_pipeline/} - Código do pipeline ETL
    \item \texttt{data/} - Dados (bronze/silver/gold/output)
    \item \texttt{docs/} - Documentação e diagramas
\end{itemize}
\end{alertblock}
\end{frame}

% ========== SECÇÃO 5: DASHBOARD ==========
\section{Dashboard Interativo}

\begin{frame}{Dashboard - Visão Geral}
\begin{block}{Funcionalidades Principais}
\begin{itemize}
    \item \textbf{Seleção de Empresa}: Dropdown com 10 empresas tech
    \item \textbf{Health Score Gauge}: Visualização do score 0-100
    \item \textbf{Métricas Principais}: Revenue, Net Income, Assets
    \item \textbf{Análise Detalhada}: Por categoria (Liquidez, Rentabilidade, etc.)
    \item \textbf{Tendências Históricas}: Evolução ao longo dos anos
    \item \textbf{Rankings Comparativos}: Comparação entre empresas
\end{itemize}
\end{block}

\begin{alertblock}{Acesso}
\textbf{URL}: \texttt{http://localhost:8501}\\
\textbf{Comando}: \texttt{streamlit run src/data\_pipeline/dashboard.py}
\end{alertblock}
\end{frame}

\begin{frame}{Dashboard - Ecrã Principal}
\begin{center}
\begin{tikzpicture}[scale=0.55, every node/.style={transform shape}]
    % Header
    \draw[thick, fill=darkblue!20] (0,9) rectangle (14,10);
    \node at (7,9.5) {\Large\textbf{Corporate Financial Health Dashboard}};
    
    % Sidebar
    \draw[thick, fill=gray!20] (0,0) rectangle (2.5,8.5);
    \node[rotate=90] at (0.3,4.5) {\textbf{SIDEBAR}};
    \node[align=center, font=\tiny] at (1.5,7) {Select\\Company};
    \draw[fill=white] (0.5,5.5) rectangle (2.3,6.5);
    \node[font=\tiny] at (1.4,6) {AAPL - Apple};
    
    % Main Content
    \draw[thick] (3,0) rectangle (14,8.5);
    
    % Status Card
    \draw[thick, fill=green!30] (3.5,7) rectangle (6.5,8);
    \node[font=\small] at (5,7.7) {\textbf{Excellent}};
    \node[font=\tiny] at (5,7.3) {Health Status};
    
    % Metrics
    \draw[thick, fill=blue!10] (7,7) rectangle (9.5,8);
    \node[font=\tiny] at (8.25,7.7) {Revenue: \$394B};
    \node[font=\tiny] at (8.25,7.3) {Growth: +2.8\%};
    
    \draw[thick, fill=blue!10] (10,7) rectangle (12.5,8);
    \node[font=\tiny] at (11.25,7.7) {Net Income: \$97B};
    \node[font=\tiny] at (11.25,7.3) {Growth: +3.2\%};
    
    % Gauge
    \draw[thick] (3.5,4.5) rectangle (7,6.5);
    \node[font=\tiny] at (5.25,6.2) {\textbf{Health Score}};
    \draw[thick] (5.25,5.5) circle (0.8);
    \node at (5.25,5.5) {\Large\textbf{85}};
    
    % Ratios Chart
    \draw[thick] (7.5,4.5) rectangle (13.5,6.5);
    \node[font=\tiny] at (10.5,6.2) {\textbf{Key Financial Ratios}};
    \foreach \x in {8,9.5,11,12.5} {
        \draw[thick, fill=green!40] (\x,5) rectangle (\x+0.5,5.8);
    }
    
    % Detailed Analysis
    \draw[thick] (3.5,0.5) rectangle (13.5,4);
    \node[font=\small] at (8.5,3.7) {\textbf{Detailed Analysis}};
    \node[align=left, font=\tiny] at (5,3) {Liquidity};
    \node[align=left, font=\tiny] at (8.5,3) {Profitability};
    \node[align=left, font=\tiny] at (12,3) {Leverage};
    
    \draw[thick, fill=green!20] (3.8,1.5) rectangle (6.5,2.5);
    \node[font=\tiny] at (5.15,2.2) {Current Ratio: 1.05};
    \node[font=\tiny] at (5.15,1.8) {Adequate liquidity};
    
    \draw[thick, fill=green!20] (7.2,1.5) rectangle (9.9,2.5);
    \node[font=\tiny] at (8.55,2.2) {Net Margin: 24.6\%};
    \node[font=\tiny] at (8.55,1.8) {Excellent profitability};
    
    \draw[thick, fill=green!20] (10.6,1.5) rectangle (13.3,2.5);
    \node[font=\tiny] at (11.95,2.2) {D/E: 2.18};
    \node[font=\tiny] at (11.95,1.8) {Moderate leverage};
\end{tikzpicture}
\end{center}
\end{frame}

\begin{frame}{Dashboard - Análise Detalhada}
\begin{block}{Análise por Categoria}
Cada métrica apresenta:
\begin{itemize}
    \item \textbf{Valor atual} e comparação com benchmark
    \item \textbf{Código de cores} (verde/amarelo/vermelho)
    \item \textbf{Interpretação textual} do resultado
    \item \textbf{Recomendações} baseadas nos valores
\end{itemize}
\end{block}

\begin{exampleblock}{Exemplo - Liquidez}
\begin{itemize}
    \item \textcolor{green}{\textbf{Current Ratio: 1.05}} - ``Adequate, but limited safety margin''
    \item \textcolor{green}{\textbf{Cash Ratio: 0.26}} - ``Adequate cash reserves''
    \item \textcolor{orange}{\textbf{Quick Ratio: 0.81}} - ``May need to liquidate inventory''
\end{itemize}
\end{exampleblock}
\end{frame}

\begin{frame}{Dashboard - Tendências Históricas}
\begin{center}
\begin{tikzpicture}[scale=0.7, every node/.style={transform shape}]
    % Revenue Trend Chart
    \draw[thick] (0,0) rectangle (6,4);
    \node[font=\small] at (3,3.7) {\textbf{Revenue Over Time}};
    \draw[->] (0.5,0.5) -- (5.5,0.5);
    \draw[->] (0.5,0.5) -- (0.5,3.5);
    \draw[thick, blue] (0.8,1) -- (1.5,1.5) -- (2.5,2) -- (3.5,2.3) -- (4.5,2.8) -- (5.2,3);
    \foreach \x/\y in {0.8/1, 1.5/1.5, 2.5/2, 3.5/2.3, 4.5/2.8, 5.2/3} {
        \fill[blue] (\x,\y) circle (0.08);
    }
    \node[font=\tiny] at (3,0.2) {Fiscal Year};
    \node[rotate=90, font=\tiny] at (0.2,2) {Revenue};
    
    % Health Score Trend Chart
    \draw[thick] (7,0) rectangle (13,4);
    \node[font=\small] at (10,3.7) {\textbf{Health Score Evolution}};
    \draw[->] (7.5,0.5) -- (12.5,0.5);
    \draw[->] (7.5,0.5) -- (7.5,3.5);
    \draw[thick, green] (7.8,2.5) -- (8.5,2.7) -- (9.5,2.8) -- (10.5,2.9) -- (11.5,3) -- (12.2,3.1);
    \foreach \x/\y in {7.8/2.5, 8.5/2.7, 9.5/2.8, 10.5/2.9, 11.5/3, 12.2/3.1} {
        \fill[green] (\x,\y) circle (0.08);
    }
    \node[font=\tiny] at (10,0.2) {Fiscal Year};
    \node[rotate=90, font=\tiny] at (7.2,2) {Score};
\end{tikzpicture}
\end{center}

\begin{alertblock}{Funcionalidades}
\begin{itemize}
    \item Gráficos interativos (hover para detalhes)
    \item Múltiplas métricas em tabs separados
    \item Dados de até 5 anos fiscais
\end{itemize}
\end{alertblock}
\end{frame}

\begin{frame}{Dashboard - Rankings}
\begin{center}
\begin{tikzpicture}[scale=0.65, every node/.style={transform shape}]
    % Revenue Ranking
    \draw[thick] (0,0) rectangle (4.5,5);
    \node[font=\small] at (2.25,4.7) {\textbf{By Revenue}};
    
    \foreach \y/\ticker/\val in {4/AAPL/\$394B, 3.4/MSFT/\$245B, 2.8/GOOGL/\$307B, 2.2/AMZN/\$575B, 1.6/NVDA/\$61B} {
        \draw[fill=blue!30] (0.3,\y-0.25) rectangle (4.2,\y-0.05);
        \node[font=\tiny, left] at (0.5,\y-0.15) {\ticker};
        \node[font=\tiny, right] at (4,\y-0.15) {\val};
    }
    
    % Health Ranking
    \draw[thick] (5,0) rectangle (9.5,5);
    \node[font=\small] at (7.25,4.7) {\textbf{By Health Score}};
    
    \foreach \y/\ticker/\val in {4/NVDA/92, 3.4/AAPL/85, 2.8/MSFT/82, 2.2/GOOGL/78, 1.6/META/75} {
        \draw[fill=green!30] (5.3,\y-0.25) rectangle (9.2,\y-0.05);
        \node[font=\tiny, left] at (5.5,\y-0.15) {\ticker};
        \node[font=\tiny, right] at (9,\y-0.15) {\val};
    }
    
    % Net Income Ranking
    \draw[thick] (10,0) rectangle (14.5,5);
    \node[font=\small] at (12.25,4.7) {\textbf{By Net Income}};
    
    \foreach \y/\ticker/\val in {4/AAPL/\$97B, 3.4/MSFT/\$72B, 2.8/GOOGL/\$74B, 2.2/NVDA/\$30B, 1.6/META/\$39B} {
        \draw[fill=yellow!40] (10.3,\y-0.25) rectangle (14.2,\y-0.05);
        \node[font=\tiny, left] at (10.5,\y-0.15) {\ticker};
        \node[font=\tiny, right] at (14,\y-0.15) {\val};
    }
\end{tikzpicture}
\end{center}

\begin{block}{Rankings Automáticos}
Sistema calcula ranks automaticamente para o ano mais recente disponível.
\end{block}
\end{frame}

% ========== SECÇÃO 6: RESULTADOS ==========
\section{Resultados e Conclusões}

\begin{frame}{Resultados Obtidos}
\begin{block}{Pipeline ETL Completo}
\begin{itemize}
    \item \textbf{10 empresas} monitorizadas
    \item \textbf{5 anos} de dados históricos por empresa
    \item \textbf{15+ KPIs} calculados automaticamente
    \item \textbf{Health Score} composto para cada empresa/ano
\end{itemize}
\end{block}

\begin{block}{Dashboard Funcional}
\begin{itemize}
    \item Interface \textbf{intuitiva} para utilizadores sem conhecimento técnico
    \item Análises \textbf{visuais} e textuais
    \item \textbf{Comparações} entre empresas
    \item \textbf{Tendências} históricas
\end{itemize}
\end{block}

\begin{alertblock}{Exemplo Real - Apple (AAPL)}
\begin{itemize}
    \item Health Score: \textbf{85/100} (Excellent)
    \item Net Margin: \textbf{24.6\%} (excelente rentabilidade)
    \item Current Ratio: \textbf{1.05} (liquidez adequada mas justa)
    \item Debt/Equity: \textbf{2.18} (alavancagem moderada)
\end{itemize}
\end{alertblock}
\end{frame}

\begin{frame}{Insights Principais}
\begin{block}{Padrões Identificados}
\begin{itemize}
    \item Empresas tech tendem a ter \textbf{margens elevadas} (20-30\%)
    \item \textbf{NVIDIA} lidera em health score (92) devido a forte crescimento
    \item \textbf{Tesla} apresenta maior volatilidade nos indicadores
    \item Empresas consolidadas (Apple, Microsoft) têm \textbf{estabilidade} superior
\end{itemize}
\end{block}

\begin{exampleblock}{Value para Decisões}
\begin{itemize}
    \item \textbf{Investidores}: Identificar empresas financeiramente saudáveis
    \item \textbf{Analistas}: Comparar performance relativa
    \item \textbf{Gestão}: Benchmark com concorrentes
    \item \textbf{Stakeholders}: Monitorizar saúde financeira ao longo do tempo
\end{itemize}
\end{exampleblock}
\end{frame}

\begin{frame}{Limitações e Trabalho Futuro}
\begin{alertblock}{Limitações Atuais}
\begin{itemize}
    \item Dados dependem da \textbf{API Yahoo Finance} (pode ter delays)
    \item Análise limitada ao \textbf{setor tecnológico}
    \item Health Score é um \textbf{modelo simplificado}
    \item Sem análise de \textbf{eventos específicos} (M\&A, reestruturações)
\end{itemize}
\end{alertblock}

\begin{block}{Melhorias Futuras}
\begin{itemize}
    \item Expandir para \textbf{múltiplos setores}
    \item Adicionar \textbf{análise de sentimento} (notícias, redes sociais)
    \item Implementar \textbf{machine learning} para previsões
    \item Integrar dados de \textbf{mercado} (cotações em tempo real)
    \item Criar \textbf{alertas} automáticos para mudanças críticas
    \item Dashboard \textbf{mobile-responsive}
\end{itemize}
\end{block}
\end{frame}

\begin{frame}{Conclusões}
\begin{block}{Objetivos Alcançados}
\begin{itemize}
    \item ✅ Sistema automatizado de extração e transformação de dados
    \item ✅ Cálculo de KPIs financeiros padronizados
    \item ✅ Dashboard interativo e user-friendly
    \item ✅ Análises comparativas entre empresas
    \item ✅ Documentação completa do projeto
\end{itemize}
\end{block}

\begin{exampleblock}{Contribuições}
\begin{itemize}
    \item \textbf{Arquitetura Medallion} aplicada a dados financeiros
    \item \textbf{Health Score} composto para avaliação rápida
    \item \textbf{Dashboard open-source} replicável
    \item \textbf{Pipeline escalável} para adicionar mais empresas/métricas
\end{itemize}
\end{exampleblock}

\begin{alertblock}{Impacto}
Ferramenta permite \textbf{democratizar análise financeira}, tornando-a acessível a utilizadores sem formação em finanças ou programação.
\end{alertblock}
\end{frame}

% ========== SLIDE FINAL ==========
\begin{frame}[plain]
\begin{center}
\vspace{2cm}
{\Huge\textbf{Obrigado!}}

\vspace{1cm}

{\Large Dashboard de Saúde Financeira}

{\large Sistema de Análise para Empresas Tecnológicas}

\vspace{1.5cm}

\begin{tabular}{ll}
\textbf{Autor:} & António Sousa \\
\textbf{Email:} & aportosousa@gmail.com \\
\textbf{Repo:} & github.com/antoniopssousa1/Corporate-Intelligence-Company-HealthStatus \\
\textbf{Dashboard:} & \texttt{streamlit run src/data\_pipeline/dashboard.py}
\end{tabular}

\vspace{1cm}

{\small FEUC - Faculdade de Economia da Universidade de Coimbra}

{\small Mestrado em Economia - Inteligência Empresarial}
\end{center}
\end{frame}

% Backup slides (opcional)
\appendix

\begin{frame}{Backup: Estrutura do Código}
\begin{block}{src/data\_pipeline/}
\begin{itemize}
    \item \texttt{config.py} - Configurações (DB, empresas, paths)
    \item \texttt{database.py} - Schema e funções DB
    \item \texttt{extract\_bronze.py} - Extração via yfinance
    \item \texttt{transform\_silver.py} - Normalização de dados
    \item \texttt{create\_gold.py} - Cálculo de KPIs
    \item \texttt{export\_excel.py} - Exports para Excel/BI
    \item \texttt{run\_pipeline.py} - Orquestração completa
    \item \texttt{dashboard.py} - Dashboard Streamlit
\end{itemize}
\end{block}
\end{frame}

\begin{frame}[fragile]{Backup: Exemplo de Código}
\begin{lstlisting}[language=Python, basicstyle=\ttfamily\tiny]
def calculate_health_score(ratios):
    """Calculate composite financial health score (0-100)"""
    score = 0
    max_score = 0
    
    # Liquidity (20 points)
    if ratios.get('current_ratio'):
        max_score += 10
        cr = ratios['current_ratio']
        if cr >= 2.0: score += 10
        elif cr >= 1.5: score += 8
        elif cr >= 1.0: score += 5
    
    # Profitability (30 points)
    if ratios.get('net_margin'):
        max_score += 15
        nm = ratios['net_margin']
        if nm >= 0.20: score += 15
        elif nm >= 0.10: score += 12
    
    # Normalize to 0-100
    return round((score / max_score) * 100, 2)
\end{lstlisting}
\end{frame}

\end{document}
