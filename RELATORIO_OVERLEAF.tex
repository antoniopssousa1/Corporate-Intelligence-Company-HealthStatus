% Relatório Técnico - Sistema de Análise Financeira de Empresas Tecnológicas
% Inteligência Empresarial - Mestrado em Engenharia e Gestão Industrial
% Universidade de Coimbra - FEUC

\documentclass[12pt,a4paper]{article}
\usepackage[utf8]{inputenc}
\usepackage[portuguese]{babel}
\usepackage{graphicx}
\usepackage{float}
\usepackage{booktabs}
\usepackage{longtable}
\usepackage{geometry}
\usepackage{hyperref}
\usepackage{listings}
\usepackage{xcolor}
\usepackage{tikz}
\usetikzlibrary{shapes.geometric, arrows, positioning, er}

\geometry{margin=2.5cm}

\title{
    \textbf{Sistema de Análise Financeira de Empresas Tecnológicas} \\
    \large Pipeline ETL com Arquitetura Medallion e Algoritmo de Avaliação de Saúde Financeira
}
\author{
    Inteligência Empresarial \\
    Mestrado em Engenharia e Gestão Industrial \\
    FEUC - Universidade de Coimbra
}
\date{Dezembro 2025}

\begin{document}

\maketitle

\begin{abstract}
Este relatório documenta o desenvolvimento de um sistema automatizado de extração, transformação e análise de dados financeiros de empresas tecnológicas cotadas em bolsa. O sistema implementa uma arquitetura Medallion (Bronze-Silver-Gold) utilizando PostgreSQL como sistema de gestão de base de dados, com um algoritmo proprietário de avaliação de saúde financeira. O objetivo final é disponibilizar dados preparados para Self-Service Business Intelligence através de ficheiros Excel e dashboards interativos.
\end{abstract}

\tableofcontents
\newpage

%==============================================================================
\section{Introdução}
%==============================================================================

\subsection{Contexto e Motivação}

A análise financeira de empresas tecnológicas apresenta desafios únicos devido à natureza dinâmica do setor e à quantidade de dados disponíveis publicamente. Este projeto surge da necessidade de criar um sistema automatizado que permita:

\begin{itemize}
    \item Extração automática de dados financeiros de múltiplas fontes
    \item Transformação e limpeza de dados para análise
    \item Cálculo de indicadores de saúde financeira
    \item Disponibilização de dados para ferramentas de Business Intelligence
\end{itemize}

\subsection{Objetivos}

\begin{enumerate}
    \item Desenvolver um pipeline ETL (Extract-Transform-Load) automatizado
    \item Implementar uma arquitetura de dados Medallion
    \item Criar um algoritmo de avaliação de saúde financeira
    \item Exportar dados prontos para visualização em Power BI/Excel
\end{enumerate}

\subsection{Empresas Analisadas}

O sistema analisa as 10 principais empresas tecnológicas por capitalização de mercado:

\begin{table}[H]
\centering
\begin{tabular}{@{}clll@{}}
\toprule
\textbf{\#} & \textbf{Ticker} & \textbf{Empresa} & \textbf{Setor} \\
\midrule
1 & AAPL & Apple Inc. & Hardware/Software \\
2 & MSFT & Microsoft Corporation & Software/Cloud \\
3 & GOOGL & Alphabet Inc. & Internet/Advertising \\
4 & AMZN & Amazon.com Inc. & E-commerce/Cloud \\
5 & NVDA & NVIDIA Corporation & Semiconductors \\
6 & META & Meta Platforms Inc. & Social Media \\
7 & TSLA & Tesla Inc. & Electric Vehicles \\
8 & AVGO & Broadcom Inc. & Semiconductors \\
9 & ASML & ASML Holding N.V. & Semiconductor Equipment \\
10 & NFLX & Netflix Inc. & Streaming \\
\bottomrule
\end{tabular}
\caption{Empresas tecnológicas analisadas}
\end{table}

%==============================================================================
\section{Arquitetura do Sistema}
%==============================================================================

\subsection{Arquitetura Medallion}

O sistema implementa a arquitetura Medallion, um padrão de design de dados que organiza a informação em três camadas progressivas de qualidade:

\begin{figure}[H]
\centering
\begin{tikzpicture}[node distance=2cm]
    % Bronze layer
    \node[draw, fill=orange!30, minimum width=10cm, minimum height=1.5cm, rounded corners] (bronze) {
        \textbf{BRONZE} - Dados em bruto (raw data)
    };
    
    % Silver layer
    \node[draw, fill=gray!30, minimum width=8cm, minimum height=1.5cm, rounded corners, above=0.5cm of bronze] (silver) {
        \textbf{SILVER} - Dados limpos e normalizados
    };
    
    % Gold layer
    \node[draw, fill=yellow!30, minimum width=6cm, minimum height=1.5cm, rounded corners, above=0.5cm of silver] (gold) {
        \textbf{GOLD} - KPIs e métricas de negócio
    };
    
    % Arrows
    \draw[->, thick] (bronze) -- (silver);
    \draw[->, thick] (silver) -- (gold);
\end{tikzpicture}
\caption{Arquitetura Medallion implementada}
\end{figure}

\subsubsection{Camada Bronze}
Armazena os dados tal como são extraídos da fonte (Yahoo Finance), sem qualquer transformação. Serve como backup histórico e permite reprocessamento.

\subsubsection{Camada Silver}
Contém dados limpos, tipados e normalizados. As tabelas seguem um modelo relacional com chaves estrangeiras e índices otimizados.

\subsubsection{Camada Gold}
Disponibiliza dados agregados, KPIs calculados e métricas prontas para consumo por ferramentas de BI.

\subsection{Stack Tecnológica}

\begin{itemize}
    \item \textbf{Linguagem}: Python 3.12
    \item \textbf{Base de Dados}: PostgreSQL 16
    \item \textbf{Extração de Dados}: yfinance (Yahoo Finance API)
    \item \textbf{Manipulação de Dados}: pandas, numpy
    \item \textbf{Conectividade BD}: psycopg2
    \item \textbf{Exportação}: openpyxl (Excel)
\end{itemize}

%==============================================================================
\section{Modelo de Dados}
%==============================================================================

\subsection{Diagrama Entidade-Relacionamento}

\begin{figure}[H]
\centering
\begin{tikzpicture}[
    entity/.style={rectangle, draw, fill=blue!20, minimum width=3cm, minimum height=1cm},
    attribute/.style={ellipse, draw, fill=white, minimum width=1.5cm, font=\small},
    relationship/.style={diamond, draw, fill=green!20, minimum width=1.5cm, aspect=2},
    every node/.style={font=\footnotesize}
]

% Silver Companies (central entity)
\node[entity] (companies) at (0,0) {\textbf{silver\_companies}};

% Bronze tables
\node[entity, fill=orange!20] (bronze_income) at (-6,3) {bronze\_income\_statement};
\node[entity, fill=orange!20] (bronze_balance) at (-6,0) {bronze\_balance\_sheet};
\node[entity, fill=orange!20] (bronze_cash) at (-6,-3) {bronze\_cash\_flow};

% Silver tables
\node[entity, fill=gray!20] (silver_income) at (0,3) {silver\_income\_statement};
\node[entity, fill=gray!20] (silver_balance) at (0,-3) {silver\_balance\_sheet};
\node[entity, fill=gray!20] (silver_cash) at (3,0) {silver\_cash\_flow};

% Gold tables
\node[entity, fill=yellow!20] (gold_health) at (6,2) {gold\_financial\_health};
\node[entity, fill=yellow!20] (gold_kpi) at (6,-2) {gold\_kpi\_dashboard};
\node[entity, fill=yellow!20] (gold_trends) at (6,0) {gold\_trends};

% Relationships
\draw[->, thick] (bronze_income) -- (silver_income);
\draw[->, thick] (bronze_balance) -- (silver_balance);
\draw[->, thick] (bronze_cash) -- (silver_cash);

\draw[-] (companies) -- (silver_income);
\draw[-] (companies) -- (silver_balance);
\draw[-] (companies) -- (silver_cash);

\draw[->, thick] (silver_income) -- (gold_health);
\draw[->, thick] (silver_balance) -- (gold_health);
\draw[->, thick] (silver_cash) -- (gold_health);

\draw[->, thick] (gold_health) -- (gold_kpi);
\draw[->, thick] (companies) -- (gold_trends);

\end{tikzpicture}
\caption{Diagrama ER simplificado do sistema}
\end{figure}

\subsection{Descrição das Tabelas}

\subsubsection{Camada Bronze}

\begin{table}[H]
\centering
\begin{tabular}{@{}lll@{}}
\toprule
\textbf{Tabela} & \textbf{Descrição} & \textbf{Campos principais} \\
\midrule
bronze\_income\_statement & Demonstração de resultados (raw) & ticker, fiscal\_year, metric\_name, metric\_value \\
bronze\_balance\_sheet & Balanço patrimonial (raw) & ticker, fiscal\_year, metric\_name, metric\_value \\
bronze\_cash\_flow & Fluxo de caixa (raw) & ticker, fiscal\_year, metric\_name, metric\_value \\
\bottomrule
\end{tabular}
\caption{Tabelas da camada Bronze}
\end{table}

\subsubsection{Camada Silver}

\begin{table}[H]
\centering
\begin{tabular}{@{}lll@{}}
\toprule
\textbf{Tabela} & \textbf{Descrição} & \textbf{Campos principais} \\
\midrule
silver\_companies & Dimensão de empresas & ticker (PK), company\_name, sector \\
silver\_income\_statement & Demonstração normalizada & revenue, net\_income, ebitda, eps \\
silver\_balance\_sheet & Balanço normalizado & total\_assets, total\_equity, current\_ratio \\
silver\_cash\_flow & Fluxo normalizado & operating\_cf, investing\_cf, free\_cf \\
\bottomrule
\end{tabular}
\caption{Tabelas da camada Silver}
\end{table}

\subsubsection{Camada Gold}

\begin{table}[H]
\centering
\begin{tabular}{@{}lll@{}}
\toprule
\textbf{Tabela} & \textbf{Descrição} & \textbf{Campos principais} \\
\midrule
gold\_financial\_health & Análise de saúde financeira & ratios, health\_score, health\_status \\
gold\_kpi\_dashboard & KPIs para dashboard & revenue, growth\%, margins, rankings \\
gold\_trends & Tendências 5 anos & cagr\_5y, trend\_direction \\
\bottomrule
\end{tabular}
\caption{Tabelas da camada Gold}
\end{table}

\subsection{Diagrama ER Detalhado}

\begin{verbatim}
┌─────────────────────────────────────────────────────────────────────────┐
│                        BRONZE LAYER                                      │
├─────────────────────────────────────────────────────────────────────────┤
│  bronze_income_statement          bronze_balance_sheet                   │
│  ├── id (PK)                      ├── id (PK)                           │
│  ├── ticker                       ├── ticker                            │
│  ├── company_name                 ├── company_name                      │
│  ├── fiscal_year                  ├── fiscal_year                       │
│  ├── metric_name                  ├── metric_name                       │
│  ├── metric_value                 ├── metric_value                      │
│  ├── raw_value                    ├── raw_value                         │
│  └── extracted_at                 └── extracted_at                      │
│                                                                          │
│  bronze_cash_flow                                                        │
│  ├── id (PK)                                                            │
│  ├── ticker                                                             │
│  ├── company_name                                                       │
│  ├── fiscal_year                                                        │
│  ├── metric_name                                                        │
│  ├── metric_value                                                       │
│  ├── raw_value                                                          │
│  └── extracted_at                                                       │
└─────────────────────────────────────────────────────────────────────────┘
                                    │
                                    ▼
┌─────────────────────────────────────────────────────────────────────────┐
│                        SILVER LAYER                                      │
├─────────────────────────────────────────────────────────────────────────┤
│  silver_companies (Dimension)                                            │
│  ├── ticker (PK)                                                        │
│  ├── company_name                                                       │
│  ├── sector                                                             │
│  └── last_updated                                                       │
│           │                                                              │
│           ├──────────────────────────────────────┐                      │
│           │                    │                 │                      │
│           ▼                    ▼                 ▼                      │
│  silver_income_statement  silver_balance_sheet  silver_cash_flow        │
│  ├── id (PK)              ├── id (PK)           ├── id (PK)             │
│  ├── ticker (FK)          ├── ticker (FK)       ├── ticker (FK)         │
│  ├── fiscal_year          ├── fiscal_year       ├── fiscal_year         │
│  ├── revenue              ├── total_assets      ├── operating_cash_flow │
│  ├── cost_of_revenue      ├── total_liabilities ├── investing_cash_flow │
│  ├── gross_profit         ├── total_equity      ├── financing_cash_flow │
│  ├── operating_income     ├── current_assets    ├── free_cash_flow      │
│  ├── net_income           ├── current_liab      ├── capex               │
│  ├── ebitda               ├── cash_equivalents  ├── dividends_paid      │
│  ├── eps_basic            ├── total_debt        └── net_change_cash     │
│  └── eps_diluted          └── retained_earnings                         │
└─────────────────────────────────────────────────────────────────────────┘
                                    │
                                    ▼
┌─────────────────────────────────────────────────────────────────────────┐
│                         GOLD LAYER                                       │
├─────────────────────────────────────────────────────────────────────────┤
│  gold_financial_health                    gold_kpi_dashboard             │
│  ├── id (PK)                              ├── id (PK)                    │
│  ├── ticker (FK)                          ├── ticker (FK)                │
│  ├── company_name                         ├── company_name               │
│  ├── fiscal_year                          ├── fiscal_year                │
│  ├── current_ratio                        ├── revenue                    │
│  ├── quick_ratio                          ├── revenue_growth             │
│  ├── cash_ratio                           ├── net_income                 │
│  ├── gross_margin                         ├── profit_growth              │
│  ├── operating_margin                     ├── health_score               │
│  ├── net_margin                           ├── health_status              │
│  ├── roe                                  ├── revenue_rank               │
│  ├── roa                                  ├── profit_rank                │
│  ├── debt_to_equity                       └── health_rank                │
│  ├── debt_to_assets                                                      │
│  ├── health_score                         gold_trends                    │
│  ├── health_status                        ├── id (PK)                    │
│  └── analysis_notes                       ├── ticker (FK)                │
│                                           ├── metric_name                │
│                                           ├── year_1..year_5             │
│                                           ├── cagr_5y                    │
│                                           └── trend_direction            │
└─────────────────────────────────────────────────────────────────────────┘
\end{verbatim}

%==============================================================================
\section{Algoritmo de Análise de Saúde Financeira}
%==============================================================================

\subsection{Metodologia}

O algoritmo de avaliação de saúde financeira analisa 5 dimensões críticas, cada uma com peso específico na pontuação final:

\begin{table}[H]
\centering
\begin{tabular}{@{}llc@{}}
\toprule
\textbf{Categoria} & \textbf{Indicadores} & \textbf{Peso} \\
\midrule
Liquidez & Current Ratio, Quick Ratio, Cash Ratio & 15\% \\
Rentabilidade & Gross Margin, Operating Margin, Net Margin, ROE, ROA & 30\% \\
Alavancagem & Debt-to-Equity, Debt-to-Assets & 20\% \\
Cash Flow & Operating CF Ratio, Free CF Margin & 20\% \\
Crescimento & Revenue Growth, Profit Growth & 15\% \\
\bottomrule
\end{tabular}
\caption{Pesos das categorias no cálculo do Health Score}
\end{table}

\subsection{Limiares de Avaliação}

Para cada indicador, foram definidos limiares baseados em benchmarks do setor tecnológico:

\begin{table}[H]
\centering
\begin{tabular}{@{}lcccc@{}}
\toprule
\textbf{Indicador} & \textbf{Excelente} & \textbf{Bom} & \textbf{Razoável} & \textbf{Fraco} \\
\midrule
Current Ratio & $\geq$ 2.0 & $\geq$ 1.5 & $\geq$ 1.0 & $<$ 1.0 \\
Net Margin & $\geq$ 20\% & $\geq$ 10\% & $\geq$ 5\% & $<$ 5\% \\
ROE & $\geq$ 20\% & $\geq$ 15\% & $\geq$ 10\% & $<$ 10\% \\
Debt-to-Equity & $\leq$ 0.5 & $\leq$ 1.0 & $\leq$ 2.0 & $>$ 2.0 \\
FCF Margin & $\geq$ 15\% & $\geq$ 10\% & $\geq$ 5\% & $<$ 5\% \\
\bottomrule
\end{tabular}
\caption{Limiares de avaliação por indicador}
\end{table}

\subsection{Classificação Final}

A pontuação final (0-100) é convertida em classificação qualitativa:

\begin{itemize}
    \item \textbf{80-100}: EXCELENTE - Fundamentos sólidos em todas as áreas
    \item \textbf{65-79}: BOM - Empresa saudável com áreas a melhorar
    \item \textbf{50-64}: RAZOÁVEL - Requer monitorização
    \item \textbf{35-49}: PREOCUPANTE - Sinais de stress financeiro
    \item \textbf{0-34}: CRÍTICO - Situação de alto risco
\end{itemize}

%==============================================================================
\section{Resultados}
%==============================================================================

\subsection{Ranking de Saúde Financeira}

\begin{table}[H]
\centering
\begin{tabular}{@{}clcll@{}}
\toprule
\textbf{Rank} & \textbf{Empresa} & \textbf{Score} & \textbf{Status} & \textbf{Principais Pontos} \\
\midrule
1 & NVIDIA & 100.0 & Excelente & Margens excecionais, crescimento >100\% \\
2 & Meta & 100.0 & Excelente & ROE 34\%, FCF Margin 33\% \\
3 & Alphabet & 94.4 & Excelente & Diversificação, baixa dívida \\
4 & Microsoft & 88.1 & Excelente & Estabilidade, cloud growth \\
5 & Netflix & 84.1 & Excelente & Turnaround em FCF \\
6 & ASML & 83.8 & Excelente & Monopólio em EUV \\
7 & Apple & 76.0 & Bom & Alta alavancagem buybacks \\
8 & Amazon & 72.6 & Bom & Margens em melhoria \\
9 & Broadcom & 71.4 & Bom & Aquisição VMware \\
10 & Tesla & 58.6 & Razoável & Margens em queda, competição \\
\bottomrule
\end{tabular}
\caption{Ranking final de saúde financeira}
\end{table}

\subsection{Análise Agregada}

\begin{itemize}
    \item \textbf{Empresas Saudáveis}: 10/10 (100\%)
    \item \textbf{Pontuação Média}: 82.9/100
    \item \textbf{Melhor Performance}: NVIDIA (crescimento de receita +114\%)
    \item \textbf{Maior ROE}: NVIDIA (91.9\%)
    \item \textbf{Menor Dívida}: NVIDIA (D/E = 0.13)
\end{itemize}

%==============================================================================
\section{Pipeline ETL}
%==============================================================================

\subsection{Fluxo de Execução}

\begin{enumerate}
    \item \textbf{Inicialização}: Criação da base de dados e tabelas
    \item \textbf{Bronze Extraction}: Download de dados via yfinance API
    \item \textbf{Silver Transformation}: Limpeza, normalização e mapeamento
    \item \textbf{Gold Aggregation}: Cálculo de KPIs e health scores
    \item \textbf{Export}: Geração de ficheiros Excel para BI
\end{enumerate}

\subsection{Estrutura de Ficheiros}

\begin{verbatim}
data_pipeline/
├── config.py              # Configurações e constantes
├── database.py            # Schema e conexões PostgreSQL
├── extract_bronze.py      # Extração via yfinance
├── transform_silver.py    # Transformações e limpeza
├── create_gold.py         # Cálculo de KPIs
├── health_analyzer.py     # Algoritmo de saúde financeira
├── export_excel.py        # Exportação para Excel
├── run_pipeline.py        # Orquestrador principal
├── bronze/                # Dados raw
├── silver/                # Dados limpos
└── gold/
    └── excel_export/      # Ficheiros para BI
        ├── MASTER_financial_data.xlsx
        ├── kpi_dashboard.xlsx
        ├── financial_health.xlsx
        └── ...
\end{verbatim}

%==============================================================================
\section{Conclusões}
%==============================================================================

\subsection{Objetivos Alcançados}

\begin{enumerate}
    \item Pipeline ETL automatizado e funcional
    \item Arquitetura Medallion implementada em PostgreSQL
    \item Algoritmo de avaliação de saúde financeira operacional
    \item Dados exportados e prontos para Power BI
\end{enumerate}

\subsection{Limitações}

\begin{itemize}
    \item Dados limitados a 5 anos históricos (limitação da API)
    \item Análise focada apenas em empresas tecnológicas
    \item Limiares calibrados para setor tech (podem não aplicar a outros setores)
\end{itemize}

\subsection{Trabalho Futuro}

\begin{itemize}
    \item Implementar atualização automática (scheduling)
    \item Adicionar mais setores e empresas
    \item Integrar com APIs de preços em tempo real
    \item Desenvolver dashboards automatizados em Power BI
\end{itemize}

%==============================================================================
\section*{Referências}
%==============================================================================

\begin{itemize}
    \item Yahoo Finance API Documentation
    \item Databricks Medallion Architecture Guide
    \item CFA Institute - Financial Ratio Analysis
    \item PostgreSQL 16 Documentation
\end{itemize}

\end{document}
